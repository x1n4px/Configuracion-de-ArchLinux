\documentclass{article}
\usepackage[letterpaper,top=2cm,bottom=2cm,left=3cm,right=3cm,marginparwidth=1.75cm]{geometry}
\usepackage{tikz}
\usetikzlibrary{automata, arrows.meta, positioning}
\usepackage{amsmath}
\usepackage{graphicx}
\usepackage[colorlinks=true, allcolors=blue]{hyperref}
\usepackage[spanish]{babel}


\begin{document}
\begin{titlepage}
\centering
\begin{figure}
\centering
{\bfseries\LARGE Escuela técnica superior\par}
\vspace{0.5cm}
{\scshape\Large Facultad de Ingeniería Informática\par}
\vspace{1cm}
\centering
\begin{Huge}
\begin{center}

\end{center}

\vspace{2cm}
\end{Huge}
\end{figure}


{\scshape\Huge Práctica Tema 3\par}
{\scshape\Large Arquitectura LTE y Handover\par}
\vspace{8cm}
{\Large Ignacio Fernández Contreras\par}
{\Large 3º Informática A\par}
\vfill

\end{titlepage}
\clearpage\hbox{}
\thispagestyle{empty}
\newpage

\newpage
\tableofcontents

\newpage


\section{Ejercicio 1. Traza S1.pcap}
\begin{flushleft}
En base a esta información, y sabiendo que estamos analizando el tráfico que fluye a través
de la interfaz S1, analiza la traza S1.pcap y responde a las siguientes preguntas añadiendo en
cada caso una captura en la que se pueda visualizar el campo o los campos en base a los
cuales se ha dado la respuesta.
\end{flushleft}
\begin{enumerate}
\item ¿Qué tipo de tráfico contiene la traza, qué está ocurriendo?\\
\begin{description}
\item \textbf{Trama 3:} Initial UE Message - Attach Request (UE$\rightarrow$ MME)
\item \textbf{Trama 4:} NAS: Auth Request (MME $\rightarrow$ UE)
\item \textbf{Trama 5:} NAS: Auth Answer (UE $\rightarrow$ MME)
\item \textbf{Trama 6:} NAS: Security Mode Command (MME $\rightarrow$ UE)
\item \textbf{Trama 7:} NAS: Security Mode Complete (UE $\rightarrow$ MME)
\item \textbf{Trama 8:} NAS: ESM Information Request (MME $\rightarrow$ UE)
\item \textbf{Trama 9:} NAS: ESM Information Answer (UE $\rightarrow$ MME)
\item \textbf{Trama 10:} S1AP: Initial Context Setup Request (MME $\rightarrow$ eNB)
\item \textbf{Trama 11:} S1AP: Initial Context Setup Response (eNB $\rightarrow$ MME)
\item \textbf{Trama 13:} NAS Attach Accept $\&$ Activate default bearer accept

\item Esto mismo se repite de la trama 22 a la 30.
\item Tras esta segunda comunicación, se encadena con:
\item \textbf{Trama 39:} GTP - Modify bearer Request (MME $\rightarrow$ SGW)
\item \textbf{Trama 40:} GTP - Modify bearer Response (SGW $\rightarrow$ MME)
\item Posterior a esto, ocurre el intercambio de información hasta la trama 1090
\end{description}










\end{document}